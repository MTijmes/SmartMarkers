\section{Projectorganisatie}
\subsection{Teamleden}
\begin{tabularx}{\textwidth}{| X | X |}
	\hline
	Dimitry Boersma & dimitry.boersma@windesheim.nl \\ \hline
	Thijs Ronda & thijs.ronda@windesheim.nl \\ \hline
	Marthijn Tijmes & marthijn.tijmes@windesheim.nl \\
	\hline
\end{tabularx}

\subsection{Afspraken}
\subsubsection{Aanwezigheid} 
Tenzij anders afgesproken wordt er samen gewerkt op Windesheim van 9:30 tot 
15:30. Wanneer het nodig is zal deze tijd verlengd worden.

\subsubsection{Te laat}
Wanneer iemand te laat komt moet diegene dit zo snel mogelijk doorgeven aan 
de rest van de groep. Ook moet diegene zich kunnen verantwoorden met een 
geldige reden.

% Wanneer iemand later aanwezig is dan de afgesproken tijd dient de persoon de 
% rest van de groepsleden hiervan te informeren. Probeer zo spoedig mogelijk 
% door te geven hoeveel tijd de vertraging bedraagt onder vermelding van de 
% reden. Gebruik de Telegram groep om iedereen te informeren.

\subsubsection{Ziekte}
Wanneer iemand ziek is dient diegene dit zo snel mogelijk door te geven aan
de rest van de groep. Als het nodig is zal de rest van de groep de werkzaamheden
van het desbetreffende persoon overnemen. 

% In het geval van ziekte dient de getroffen persoon het team te informeren voor 
% aanvang van het eerst lesuur op de eerstvolgende lesdag (maandag t/m vrijdag). 

\subsubsection{Overig}
Als iemand niet in staat is om bij het project aanwezig te zijn dient dit doorgegeven
te worden. Hierbij moet ook worden vermeldt wanneer deze persoon weer aanwezig kan 
zijn.

% Als een deelnemer weet dat hij/zij afwezig is op één of meer dag(en) moet dit 
% besproken worden met het team. Dit dient minimaal twee dagen voor de 
% afwezigheid worden besproken om de planning aan te kunnen passen. Om verwarring 
% te voorkomen dient de deelnemer de afwezigheid ook aan te kondigen in de 
% Telegram groep.
% In het geval van persoonlijke omstandigheden is het niet verplicht om een reden 
% op te geven voor de afwezigheid als de deelnemer dit niet wenst op te geven.
