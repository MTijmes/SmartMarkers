% Pad with two zeroes
\newcommand\padzeroes[1]{\ifnum #1 < 10 0\fi\ifnum #1 < 100 0\fi #1}

% This stuff does automatic user story numbering
\newcommand{\ustorylbl}{FREQ}
\newcounter{storynr}
\newcommand{\story}{\subsubsection{\ustorylbl-\padzeroes{\arabic{storynr}}}}
\stepcounter{storynr}

\section{Eisen}
\subsection{User Stories}
\story
Als Kadaster medewerker wil ik de locatie van de SmartMarkers kunnen zien via
een kaart.
\stepcounter{storynr}

\story
Als Kadaster medewerker wil ik de SmartMarkers kunnen identificeren.
\stepcounter{storynr}

\story
Als Kadaster medewerker wil ik kadastrale registratie kunnen doen,
met behulp van de SmartMarkers kaart.
\stepcounter{storynr}

\story
Als SmartMarker plaatser wil ik kunnen zien dat een SmartMarker aanstaat.
\stepcounter{storynr}

\story
Als SmartMarker plaatser wil ik kunnen zien dat een SmartMarker verbonden
is met het netwerk.
\stepcounter{storynr}

\story
Als SmartMarker plaatser wil ik geen systeem instellingen hoeven aan te passen.
\stepcounter{storynr}

\story
Als SmartMarker plaatser wil ik weinig onderhoud hoeven te doen aan de
SmartMarker (Batterij duur).
\stepcounter{storynr}

\newpage
\subsection{Milestones}
\begin{itemize}
    \item Een prototype van een enkel SmartMarker dat zijn locatie kan bepalen
    \item Een tweede SmartMarker dat met het eerste SmartMarker kan communiceren
    \item Vijf SmartMarkers die met elkaar kunnen communiceren
    \item Een weergave van de locatie van de verschillende SmartMarkers op
          een kaart
\end{itemize}
\subsection{Definition of Done}
Iets is pas klaar als:
\begin{itemize}
    \item De code geschreven is volgens de code standaard die in het plan
        van aanpak beschreven is.
    \item De code volledig is getest volgens ons testplan.
    \item De code ook unittests heeft waar het nodig is.
    \item De documentatie klopt met de code
    \item De code is gereviewd door tenminste één projectgenoot
\end{itemize}
