% Pad with two zeroes
\newcommand\padzeroes[1]{\ifnum #1 < 10 0\fi\ifnum #1 < 100 0\fi #1}

% This stuff does automatic user story numbering
\newcommand{\ustorylbl}{SMRK}
\newcounter{storynr}
\newcommand{\story}{\subsubsection{\ustorylbl-\padzeroes{\arabic{storynr}}}}
\stepcounter{storynr}

\section{Eisen}
\subsection{User Stories}
%\subsubsection{\story}
\story
Als gebruiker wil ik de locatie van een paaltje kunnen zien via een kaart
\stepcounter{storynr}

\story
Als basisstation wil ik de locatie van de verschillende paaltjes kunnen 
ontvangen (te technisch?)
\stepcounter{storynr}

\story
Als gebruiker wil ik het ID van een paaltje kunnen zien in een overzicht?
\stepcounter{storynr}

\story
Als gebruiker wil ik van alle verschillende paaltjes de locatie kunnen zien
op een kaart
\stepcounter{storynr}

\story
Als gebruiker wil ik kunnen zien dat een paaltje aanstaat
\stepcounter{storynr}

\story
Als gebruiker wil ik kunnen zien dat een paaltje verbonden is met het (mesh)
netwerk
\stepcounter{storynr}

\subsection{Milestones}
\begin{itemize}
    \item Een prototype van een enkel paaltje dat zijn locatie kan bepalen
    \item Een tweede paaltje dat met het eerste paaltje kan communiceren
    \item Vijf paaltjes die met elkaar kunnen communiceren
    \item Een weergave van de locatie van de verschillende paaltjes op een kaart
\end{itemize}
\subsection{Definition of Done}
