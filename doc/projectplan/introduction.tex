\section{Inleiding}
In Nederland is het vanzelfsprekend dat vastgoed geregistreerd is en dat er bekend
is welke gegevens bij dat vastgoed horen. Het Kadaster is verantwoordelijk voor de
vastlegging en verstrekking van vastgoed- en geografische informatie en de rechten
die daarbij horen, zoals eigendom en hypotheek. \citep{KAD_OVER}

Nederland is op gebied van kadastrale registratie vooruitstrevend; in het buitenland
is het niet vanzelfsprekend dat gegevens van vastgoed ergens zijn vastgelegd.
Als gevolg daarvan kan men in deze landen niet aantonen wat hun ontroerend goed is
of wat de waarde daarvan is.

Uit de morele overtuiging dat iedereen recht op eigendom heeft, wat onder andere
opgenomen is in artikel 17 van de Universele Verklaring van de Rechten van
de Mens (UVRM) \citep{UN_UDHR}, wil het Kadaster International deze landen een systeem bieden waarmee
deze eigendommen in kaart gebracht kunnen worden. Dit systeem is recentelijk in
ontwikkeling genomen en het project draagt de naam: \textit{Smart Markers}

\subsection{Aanleiding}
Door het Internet-of-Things keuzesemester van de opleiding hbo-ICT, te Windesheim,
zijn de uitvoerende studenten in aanraking gekomen met dit project. Tijdens het
keuzesemester is er een periode van tien weken waarin studenten aan een project
werken wat in het kader staat van (inter)connectiviteit, duurzaamheid en in het
belang is van de maatschappij.

\subsection{Doel}
Het doel van het project is om ontwikkelingslanden een systeem te bieden dat
een soortgelijke werking heeft als de kadastrale registratie van het Kadaster
in Nederland.

Binnen de scope van tien weken betekent dit het ontwikkelen van een prototype
van een systeem waar locatiegegevens mee kunnen worden verzameld. In hoofdstuk
\ref{sec:opdracht} wordt dit systeem verder uitgediept.
