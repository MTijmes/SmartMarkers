\section{Systeemarchitectuur}
De architectuur van het Smart Markers systeem is een netwerk van GPS-nodes die communiceren met de gateway middels LoRa om tot op een meter nauwkeurig locatiegegevens te registreren. De end-nodes verzamelen de GPS-data en versturen deze naar de gateway. De gateway, ookwel het referentie- of basisstation, berekent een GPS-correctie en voert deze uit op de data van de end-nodes.

\subsection{Architectuur}
In hoofdstuk \ref{subsec:overzicht} is een schematische weergave van de Smart Markers architectuur.
% TODO: Welke protocollen? Welke data?

\subsection{Hardware}
In de volgende paragrafen staan beschrijvingen van de gebruikte hardware.
\paragraph{STM32 microcontroller}
Als brein van de Smart Marker is er gekozen voor een \texttt{STM32 NUCLEO-L073RZ development board}, omdat het board:
\begin{itemize}
    \item beschikt over een ARM Cortex M0+ 32 bit processor;
    \item beschikt over voldoende geheugen voor de Smart Marker applicatie;
    \item weinig energie verbruikt;
    \item zich leent voor prototyping;
    \item een veelgebruikte en bekende architectuur heeft;
\end{itemize}

\paragraph{LoRa radio}
De LoRa radio bevindt zich op het RF-LORA-868-SO board. De SX1272 LoRa radio wordt voornamelijk gebruikt voor end-devices, omdat het een half-duplex radio is.
% TODO: Links naar LoRa docs

\paragraph{Locatiebepaling}
Voor de locatiebepaling worden u-blox EVK-7P GPS-modules gebruikt, omdat de opdrachtgever al in bezit was van deze GPS-modules.
% TODO: Meer onderzoeken. Welke data?

\subsection{Software}
Tijdens het project wordt er gebruik gemaakt van softwarepakketten ter ondersteuning. Hieronder staat een korte opsomming van tools en technieken die tijdens het project worden gebruikt:
\begin{itemize}
    \item GNU ARM embedded toolchain;
    \item Makefiles;
    \item ST-Link programmer;
    \item Git;
    \item \LaTeX;
\end{itemize}

\subsection{Communicatie}
Als communicatie techniek is er gekozen voor LoRa, omdat de opdrachtgever met één meting een heel gebied wilt kunnen registreren. De LoRa modulatietechniek leent zich ervoor om in deze use-case gebruikt te worden, vanwege de lange communicatie-afstand en het lage energieverbruik.
% TODO: Bron toevoegen vergelijking communicatie technieken/protocollen

Er is overwogen om een LoRa mesh netwerk op te zetten, echter is dat achterwege gelaten vanwege de korte tijdsduur van het project. Uit het onderzoek van Kai-Hsiang Ke et al. is gebleken dat de packet delivery ratio (PDR) van hun labopstelling aanzienlijk is verbeterd door gebruik te maken van een mesh netwerk.
% Bron: http://mx.nthu.edu.tw/~huclee/paper/2017%20IPSN%20LORA%20MESH%2020170124.pdf

\subsection{Gegevensopslag}
Beschrijving van de manier(en) van gegevensopslag.
% TODO: Uitzoeken wat de data throughput is.

\label{subsec:overzicht}
\subsection{Overzicht}
Een totaaloverzicht (afbeelding) van de systeemarchitectuur.
