\section{Gegevensmodel}
\subsection{Data format}
\subsection{Database}
\label{sec:database}

De verstuurde data zal worden bijgehouden in een SQL-database. Tijdens het testen
zal dit op een extern gehoste server zijn.
Wanneer er een eigen LoRa gateway met de markers gebruikt wordt zal deze database
een lokaal, offline, systeem zijn. De data zal dan naar een centrale database
worden verstuurd wanneer er een goede internetverbinding is.

De tabel met de meetpunten zal de volgende kolommen bevatten:
\begin{itemize}
    \item Meetpunt ID
    \item Marker nummer
    \item Latitude
    \item Longitude
    \item Tijdstip van registratie
\end{itemize}

Verder zal er een tabel zijn met de percelen, deze bevat de kolommen:
\begin{itemize}
    \item Perceel ID
    \item Perceelnummer
\end{itemize}

Om de percelen en meetpunten aan elkaar te koppelen zal er een koppeltabel komen:
\begin{itemize}
    \item Perceel ID
    \item Volgnummer
    \item Meetpunt ID
\end{itemize}

\subsection{Bestanden}
\subsubsection{Code}
