\section{Gebruikersinterface}
\subsection{Dataweergave}
De verzonden meetpunten zullen worden getoond op een website met Google Maps
kaartmateriaal waarop de meetpunten getoond worden door middel van KML-imports.

\subsection{Advies}
De Google Maps plugin is afhankelijk van internet, dat wil zeggen dat er een actieve
verbinding moet zijn om het kaartmateriaal te downloaden. Dit kan in het veld een
probleem zijn.

Het tonen van de polygonen is binnen KML-bestanden wel mogelijk, het is echter nog
wel uitzoeken hoe dit het makkelijkst gedaan kan worden.
