\section{Gebruikersinterface}
\subsection{Dataweergave}
Voor het weergeven van de meetpunten is gekozen voor een JavaScript plugin van Google.
Op deze manier laat zich eenvoudig een Google Maps kaart tonen op de website waarop we de
meetpunten en percelen eenvoudig kunnen tonen.

Het importeren van de meetpunten wordt gedaan door de meetpunten te exporteren naar een
KML-bestand, welke de coordinaten en overige gegevens van de punten bevat. Deze worden
door een PHP-script uit de SQL-database gehaald en vervolgens geprint in het KML-bestand.
Deze worden vervolgens als pins binnen Google Maps getoond. Om te voorkomen dat er een
door Google gecachte versie van het KML-bestand getoond wordt, wordt de time() functie van
php gebruikt om bij elke aanroep een nieuwe URL te creëren.

Door op een pin te klikken wordt de bijbehorende informatie getoond, samen met een 'Voeg
aan perceel toe' button, om de geselecteerde marker toe te voegen aan een nieuw perceel.

Op de achtergrond maakt een Javascript functie hier een array van, welke samen met de
ingevoerde naam verzonden wordt naar de server.
Wanneer er een perceel verzonden wordt, maakt de database hier eerst een perceel ID voor
aan, vervolgens worden alle meetpunten in de array met het nieuwe perceel ID toevevoegd
aan de link\_table.

De percelen worden ook verwerkt tot een KML-bestand. Hierin wordt per perceel een nieuwe
polygoon aangemaakt, waarin op volgorde van aanklikken de meetpunten als hoekpunten
toegevoegd worden. Dit wordt gedaan door een automatisch verhogend ID veld te hebben in de
link\_table en hierop ook weer te sorteren.

Wanneer op een polygoon geklikt wordt, toont de website enige informatie over dit perceel.
Op dit moment de naam en het tijdstip van creëren, maar dit is eenvoudig aan te passen naar
meer informatie indien gewenst.


\subsection{Advies}
De Google Maps plugin is afhankelijk van internet, dat wil zeggen dat er een actieve
verbinding moet zijn om het kaartmateriaal te downloaden. Dit kan in het veld een
probleem zijn.
