\section{Reflectie}
\subsection{Rick de Bondt}
Ik heb gewerkt aan het uitlezen en instellen van ge GPS-module. Verder heb ik
gewerkt aan het herschrijven van I2C, SPI en RTC. Het herschrijven van SPI is
uiteindelijk overgenomen door Jerko.

Het grootste probleem waar ik tegen aanliep in dit project was vooral het
herschrijven van de bibliotheken. Ik heb veel in de documentatie en code moeten
zoeken voordat ik in de gaten had hoe bijvoorbeeld I2C werkt op een STM32.
Vervolgens moest het dan nog werken met de bestaande LoRa-code. Dit heeft veel
tijd gekost. Ik verwacht dat het beter was gelopen als we een andere LoRa-module
hadden besteld.

Tegen het einde was ik helemaal vastgelopen op de RTC. De timers
die gezet kunnen worden in de LoRa-code wilden het maar een paar keer doen. Dit
vond ik lastig, omdat we de RTC nodig hadden om het project volledig werkend te
maken.

Qua tijd bijhouden is het aardig gelukt, maar vooral in de laatste weken is het
er door achterstand nog wel eens bij in geschoten.

Voor de documentatie heb ik me vooral bezig gehouden met het functioneel ontwerp
en de onderdelen van het technisch ontwerp waaraan ik gewerkt heb. Dit houd
vooral de onderdelen gerelateerd aan GPS in. Dit had ik beter bij kunnen houden,
de informatie zelf ben ik aardig tevreden over.

\subsection{Jerko Lenstra}
Het Smart Markers project vond ik leuk en uitdagend om aan te werken, maar het
project was van veel te korte duur om een goed product neer te kunnen zetten.
We hebben behoorlijk wat moeten schrappen van de projectplanning, omdat er gewoon
geen tijd voor was om het te implementeren. Het leek mij erg leuk om aan de gang
te gaan met een LoRa mesh netwerk te onderzoeken en op te bouwen.

Omdat ik eerder met LoRa heb gewerkt verwachtte ik dat het implementeren van LoRa
erg simpel zou zijn en niet veel tijd in beslag zou nemen, daarom leek het mij
handig voor de gebruiker (Kadaster) om een abstractie laag tussen de LoRaWAN stack
en de gebruiker applicatie te hebben. Dit heeft veel langer geduurd dan verwacht,
en heeft ervoor gezorgd dat op sommige momenten mijn teamgenoten op mij moesten
wachten voordat zij verder konden.

Het KBS is niet altijd even soepel verlopen, omdat geen van ons echte leiderschap
karakteristieken heeft, daardoor was het een uitdaging om op de juiste koers te
blijven, maar volgens mij is het redelijk gelukt. Los van een bug met de timers
die we niet hebben kunnen vinden, hebben we ons doel bereikt en heb ik een hoop
geleerd.

Het was interessant om voor een echt bedrijf een opdracht te mogen doen, maar het
bracht mij ook enige stress mee, omdat de communicatie tussen de opdrachtgever en
ons niet altijd heel helder is geweest. We zijn aan het begin van het project best
veel tijd verloren in het afstemmen van wat de opdracht precies is. Achteraf denk
ik dat het minder chaotisch was geweest als we onze eigen opdracht hadden verzonnen.

\subsection{Marthijn Tijmes}
