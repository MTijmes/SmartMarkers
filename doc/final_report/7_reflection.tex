\section{Reflectie}

\subsection{Rick de Bondt}
Ik heb gewerkt aan het uitlezen en instellen van ge GPS-module. Verder heb ik
gewerkt aan het herschrijven van I2C, SPI en RTC. Het herschrijven van SPI is
uiteindelijk overgenomen door Jerko.

Het grootste probleem waar ik tegen aanliep in dit project was vooral het
herschrijven van de bibliotheken. Ik heb veel in de documentatie en code moeten
zoeken voordat ik in de gaten had hoe bijvoorbeeld I2C werkt op een STM32.
Vervolgens moest het dan nog werken met de bestaande LoRa-code. Dit heeft veel
tijd gekost. Ik verwacht dat het beter was gelopen als we een andere LoRa-module
hadden besteld.

Tegen het einde was ik helemaal vastgelopen op de RTC. De timers
die gezet kunnen worden in de LoRa-code wilden het maar een paar keer doen. Dit
vond ik lastig, omdat we de RTC nodig hadden om het project volledig werkend te
maken.

Qua tijd bijhouden is het aardig gelukt, maar vooral in de laatste weken is het
er door achterstand nog wel eens bij in geschoten.

Voor de documentatie heb ik me vooral bezig gehouden met het functioneel ontwerp
en de onderdelen van het technisch ontwerp waaraan ik gewerkt heb. Dit houd
vooral de onderdelen gerelateerd aan GPS in. Dit had ik beter bij kunnen houden,
de informatie zelf ben ik aardig tevreden over.
