\section{Security}
\textbf{Het verplaatsen van de marker voordat deze zijn locatie doorgestuurd
heeft} \newline
Paul Saers van het Kadaster gaf aan dat dit geen probleem was; hij gaat er
vanuit dat de markers niet verplaatst worden. Hij verwacht dat de goodwill van
de lokale bevolking groot genoeg is om de metingen niet te saboteren.

Eventueel is hier wel op te controleren met een bewegingssensor, waardoor
meetpunten niet verzonden worden of bijvoorbeeld een kenmerk krijgen dat deze
mogelijk onbetrouwbaar zijn.

\textbf{Het onderscheppen van verzonden gegevens} \newline
Gezien de Smart Markers naar alle waarschijnlijkheid op een eigen gateway gaan
werken en het KPN netwerk momenteel alleen gebruikt wordt om te testen, is hier
nog niet veel aandacht aan besteed.

Wel wordt het verkeer versleuteld naar de portal van KPN gestuurd en via een
HTTPS verbinding naar de server, wat de veiligheid al een heel stuk waarborgd.
De server is voorzien van certificaten, op dit moment van Let's Encrypt.org

\textbf{Het inzien van de opgeslagen data} \newline
Op dit moment draait de database op een vrij te benaderen webserver bij één van
de projectleden thuis. De Smart Markers zullen naar alle waarschijnlijkheid een
lokale server bij de gateway hebben om de data op te slaan, welke dan niet vanaf
het internet te benaderen is.

Wel is het natuurlijk mogelijk om de website af te schermen met een inlog, zodat
het niet mogelijk is om zomaar de data in te zien. Voor het gemak van het
project en gezien het feit dat we nu nog geen gevoelige data verwerken hebben we
hier nog niet voor gekozen.

Daarnaast is de veiligheid van de database te vergroten door enkel vanaf
localhost (de server zelf), eventueel aangevuld met een whitelist van eigen
IP-adressen, inloggen toe te staan. Op dit moment is echter inloggen vanaf elke
locatie mogelijk, gezien de vele plaatsen vanaf waar er gewerkt wordt aan dit
project. In een productieomgeving is dit zwaar af te raden!

\textbf{Het verlies van opgeslagen data} \newline
Peter Brouwer van het Kadaster gaf aan dat eigenlijk overal een (goede)
internetverbinding beschikbaar kan zijn. Het idee was om de lokaal verzamelde
data, wanneer het eenmaal compleet is, te versturen naar een externe bron.
Daarnaast kan het natuurlijk ook lokaal dubbel opgeslagen worden of gebackupt
worden.

\textbf{Onnauwkeurigheid van de meting} \newline
De GPS-ontvangers zelf hebben een indicatie van hoe nauwkeurig ze zijn. Er kan
dus worden besloten om nog niet uit te zenden wanneer deze een (grote) afwijking
heeft. Daarnaast verwachten we dat een langer geplaatste marker, met meerdere
uitgevoerde metingen, ook een betrouwbaarder gemiddelde zal geven.
