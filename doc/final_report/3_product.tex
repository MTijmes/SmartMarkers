\section{Product}
Momenteel kan het prototype de locatie uit de GPS halen, deze verzenden naar de
database en deze kan getoond worden op de kaart. Verder zijn van de weergegeven
markeringen percelen te maken.

In deze ontwikkelfase wordt er nog gebruik gemaakt van het KPN LoRa netwerk, wat
natuurlijk in de doelgebieden niet beschikbaar is. Daarom zal er daar met een
eigen gateway en lokale server gewerkt moeten worden om de GPS-data van de Smart
Markers te verzamelen en verwerken.

Daarnaast kan de energievoorziening gerealiseerd worden, op dit moment is het
prototype alleen in gebruik geweest met een laptop. Te denken valt aan een
(interne), oplaadbare, stroomvoorziening om de marker makkelijker in het veld te
kunnen plaatsen.

Verder hebben we het met het Kadaster gehad over manieren om de percelen met
minder handwerk te maken, bijvoorbeeld door het analyseren van sattelietbeelden.
