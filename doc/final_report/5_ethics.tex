\section{Ethische aspecten}
\subsection{Duurzaamheid}
Wat betreft de duurzaamheid hebben we gekozen voor een STM variant welke in een
heel zuinige sleepmode gezet kan worden. Op deze manier zal er zo min mogelijk
energie verbruikt gaan worden door de Smart Markers.

Voor de energievoorziening werd er aan oplaadbare 18650 cellen gedacht. De
markers komen met grote regelmaat weer terug uit het veld, waardoor het erg
simpel is deze cellen tussentijds op te laden. Dit kan uiteraard met groene
energie.

Het idee is dat de markers los van hun voet gehaald kunnen worden, zodat deze,
indien gewenst, achter kan blijven als markering. De marker zelf is meerdere
keren te gebruiken en zal dus vele metingen uit kunnen voeren. Daardoor hoeven
er relatief weinig gemaakt te worden.

\subsection{Welzijn}
In de Universele Verklaring van de Rechten van de Mens (UVRM) van de Verenigde
Naties staat dat iedereen recht heeft op eigendom. Het eigendom van de grond, en
dat op papier bevestigd hebben, draagt daar natuurlijk enorm aan bij.
Bij geschillen en conflicten kan het enorm helpen om duidelijk te hebben waar
grenzen lopen en wie welk perceel in eigendom heeft. Dit is nu vaak al wel
lokaal bekend maar het centraliseren van deze informatie zal naar verwachting
zeker een positieve invloed hebben.

Uiteraard is het natuurlijk niet uit te sluiten dat er conflicten ontstaan door
het in kaart brengen van grenzen, maar omdat de grenzen al bestaan werd verwacht
dat dit geen grote problemen zou opleveren.

\subsection{Gezondheid}
Wat betreft de gezondheid zal de Smart Marker weinig tot geen invloed hebben.
Natuurlijk komen er op een relatief klein gebied kortdurig vele markers te
staan, maar het idee is dat deze, samen met de gateway, na enkele dagen weer
verplaatst worden naar een volgend gebied. De elektromagnetische straling zal,
voor zo ver dit überhaupt invloed heeft, daarom nihil zijn gezien de
tijdsperiode dat de mensen in het gebied hieraan blootgesteld worden.
