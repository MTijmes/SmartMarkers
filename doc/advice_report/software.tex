\section{Software}

\subsection{Verbeteren nauwkeurigheid meetpunten}
In de huidige versie wordt het DGPS-signaal nog niet optimaal gebruikt en verwerkt.
Om de nauwkeruigheid van de meetpunten te verbeteren zal hier nog meer onderzoek
naar gedaan moeten worden hoe dit precies toe te passen binnen het prototype.

\subsection{Gegevensopslag}
Op dit moment worden de gegevens opgeslagen in een SQL-database. Het is echter
niet onderzocht of dit de meest ideale manier van opslag is. Zowel voor de grootte
en het type relaties alsmede de koppeling met eventuele bestaande systemen kan
nog gekeken worden naar een alternatief.

Daarnaast speelt de beveiliging van de opgeslagen data natuurlijk een grote rol.
Momenteel is er geen beveiliging, gezien we op dit moment alleen nog maar dummy data
verwerkt hebben. Wanneer er echte meetpunten verwerkt gaan worden is het beveiligen
hiervan natuurlijk wel belangrijk.

\subsection{Automatiseren aanmaken percelen}
Tijdens één van de gesprekken met het Kadaster kwam het idee naar voren om
satellietbeelden te gebruiken om (een deel van de grenzen) automatisch te verwerken.
Dit is nog helemaal niet onderzocht gezien dit ver van de embedded systems staat.
Wellicht is dit voor programmeurs met meer ervaring in deze richting nog een
interessant project.
