\section{Hardware}

\subsection{Netwerk}
Voor de ontwikkeling van het prototype is er gebruik gemaakt van het landelijk
dekkende LoRa-netwerk van KPN, maar voor doorontwikkeling en plaatsing in het
doelgebied is het noodzakelijk om een privaat netwerk te implementeren.
Zowel het implementeren van een privaat netwerk als het uitvoerig onderzoeken
ervan viel buiten de scope van dit project. \\
Tijdens de uitvoering van dit project zijn er bevindingen m.b.t. LoRaWAN
netwerken gedaan, deze zijn vertaald naar het onderstaande advies:

\paragraph{Vervolgonderzoek: sternetwerk}
Het netwerk met stertopologie is wat LoRaWAN standaard ondersteunt, dit betekent
dat er voor deze implementatie veel informatie beschikbaar is. \citep{LRSRV} \\
Concreet betekent het implementeren van een sternetwerk dat er minder
ontwikkeltijd benodigd is om het Smart Markers systeem te realiseren.

\paragraph{Vervolgonderzoek: mesh-netwerk}
Het LoRaWAN protocol ondersteunt geen mesh-netwerken. Echter, uit onderzoek is
gebleken dat het gebruik van LoRa (lees: de modulatietechniek) in een netwerk
met mesh-topologie een hogere \textit{packet delivery ratio}
\citep{AIM_HI_LORA_WMN} heeft dan een netwerk met stertopologie. Over het
implementeren van LoRa in een mesh-netwerk is weinig informatie beschikbaar,
waarschijnlijk omdat er doorgaans meer gebruik gemaakt wordt van LoRaWAN en
vanwege de langere ontwikkeltijd. \\
Concreet betekent het implementeren van een mesh-netwerk voor hogere robuustheid
van het netwerk, maar ook dat er (aanzienlijk) meer onderzoek gedaan moet worden
naar onder andere:
\begin{itemize}
    \item het ontwikkelen van een mesh-protocol voor LoRa;
    \item het oplossen van radio timing issues (asynchrone communicatie zal
          zorgen voor meer packet collisions);
    \item of \textit{backhaul} nodes benodigd zijn (lees: nodes rond een gateway
          of server die veel netwerkverkeer afhandelen);
\end{itemize}

\paragraph{Advies}
Om op korte termijn het Smart Markers systeem te realiseren luidt het advies om
gebruik te maken van LoRaWAN nodes in een sternetwerk opstelling. Als uit de
ervaring blijkt dat een sternetwerk zorgt voor teveel onvoorspelbaarheid (packet
collisions of packet loss) is het raadzaam om een robuuster netwerk te
realiseren d.m.v. het implementeren van een mesh-netwerk.

\subsection{Energievoorziening}
Op dit moment is het prototype alleen in gebruik geweest met een laptop. Daardoor
is er nog niet serieus gekeken naar voeding om de markers in het veld te plaatsen.

Er is gedacht aan 2 18650 cellen, welke zo'n 3.5V per stuk leveren. Deze zijn
oplaadbaar en hebben een vrij hoge capaciteit. Daarnaast zijn deze makkelijk te
vervangen.

Het opladen zou kunnen geschieden door een zonnepaneel toe te voegen of. Of, gezien
de markers toch regelmatig terugkomen, ze in een bewaarkrat te prikken met connectoren
om daarin op te laden.

\subsection{Server}
Omdat we tijdens de ontwikkelfase gebruik hebben gemaakt van het KPN LoRa netwerk, was
het verzenden naar een online server mogelijk.
Tijdens het meten in de ontwikkelingslanden zal dit niet zo zijn, dus zal er een lokale
server meegenomen moeten worden naar de gebieden. Dit kan natuurlijk ook een laptop zijn.

Ook komt hier nog enige beveiliging bij kijken, zowel wat back-ups als toegang tot
gegevens betreft. Tijdens de gesprekken met het Kadaster werd duidelijk dat het goed
mogelijk was om regelmatig over (goed) internet te beschikken om de lokale data te
versturen naar een andere server.

\subsection{GPS}
De GPS-module die we gebruikt hebben tijdens het project is de uBlox Neo-7P.
Deze module werkt naar verwachting. Voor een nauwkeurigheid van ongeveer een
meter is een module die DGPS ondersteund vereist. De prijs van de uBlox Neo-7P
is alleen hoger dan gewenst is. Mocht er een goedkoper alternatief uitkomen dan
gaat de voorkeur daar naar uit. 

Tijdens het realiseren van het project is het ons niet gelukt om DGPS werkend te
krijgen. Voor een nauwkeurige meting is het van belang DGPS in de code te
implementeren.

\subsection{Behuizing}
Hoewel er wel over gepraat is, is er nog niet gekeken naar een behuizing van de
Smart Marker. Dit kan echter pas afgerond worden als de andere keuzes vast staan,
omdat het gebruiken van een andere chip en de keuze van de stroomvoorziening hier
grote invloed op hebben.
