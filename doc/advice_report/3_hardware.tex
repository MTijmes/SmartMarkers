\section{Hardware}

\subsection{Netwerk}
In deze ontwikkelfase wordt er nog gebruik gemaakt van het KPN LoRa netwerk, wat
natuurlijk in de doelgebieden niet beschikbaar is. Daarom zal er daar met een eigen
gateway en lokale server gewerkt moeten worden om de GPS-data van de Smart Markers
te verzamelen en verwerken. Een andere optie is het werken met een mesh-netwerk.

\subsection{Energievoorziening}
Op dit moment is het prototype alleen in gebruik geweest met een laptop. Daardoor
is er nog niet serieus gekeken naar voeding om de markers in het veld te plaatsen.

Er is gedacht aan 2 18650 cellen, welke zo'n 3.5V per stuk leveren. Deze zijn
oplaadbaar en hebben een vrij hoge capaciteit. Daarnaast zijn deze makkelijk te
vervangen.

Het opladen zou kunnen geschieden door een zonnepaneel toe te voegen of. Of, gezien
de markers toch regelmatig terugkomen, ze in een bewaarkrat te prikken met connectoren
om daarin op te laden.

\subsection{Server}
Omdat we tijdens de ontwikkelfase gebruik hebben gemaakt van het KPN LoRa netwerk, was
het verzenden naar een online server mogelijk.
Tijdens het meten in de ontwikkelingslanden zal dit niet zo zijn, dus zal er een lokale
server meegenomen moeten worden naar de gebieden. Dit kan natuurlijk ook een laptop zijn.

Ook komt hier nog enige beveiliging bij kijken, zowel wat back-ups als toegang tot
gegevens betreft. Tijdens de gesprekken met het Kadaster werd duidelijk dat het goed
mogelijk was om regelmatig over (goed) internet te beschikken om de lokale data te
versturen naar een andere server.

\subsection{GPS}
De GPS-module die we gebruikt hebben tijdens het project is de uBlox Neo-7P.
Deze module werkt naar verwachting. Voor een nauwkeurigheid van ongeveer een
meter is een module die DGPS ondersteund vereist. De prijs van de uBlox Neo-7P
is alleen hoger dan gewenst is. Mocht er een goedkoper alternatief uitkomen dan
gaat de voorkeur daar naar uit. 

Tijdens het realiseren van het project is het ons niet gelukt om DGPS werkend te
krijgen. Voor een nauwkeurige meting is het van belang DGPS in de code te
implementeren.

\subsection{Behuizing}
Hoewel er wel over gepraat is, is er nog niet gekeken naar een behuizing van de
Smart Marker. Dit kan echter pas afgerond worden als de andere keuzes vast staan,
omdat het gebruiken van een andere chip en de keuze van de stroomvoorziening hier
grote invloed op hebben.
