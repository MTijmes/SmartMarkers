\section{Kwaliteit}
Bij de start van het project worden codestandaarden opgesteld. Om de kwaliteit
van het project te waarborgen is iedere deelnemer verantwoordelijk voor de door
hem/haar opgeleverde producten. Deze kunnen in overeenstemming met andere
teamleden als kwalitatief onvoldoende worden beschouwd. In dit geval is de
persoon verantwoordelijk voor het wijzigen van het product om de kwaliteitseisen 
te behalen. Om de kwaliteit van de code te waarborgen wordt gebruikt gemaakt van 
unittesting.

\subsection{Versiebeheer}
\label{sec:versiebeheer}

Het versiebeheer is geregeld met een Git repository gehost op GitHub.

\subsection{Git}
\subsubsection{Forks and branches}
Om samen te werken aan een project in een Git repository wordt er gebruik
gemaakt van forks and branches. Middels pull requests naar de develop branch
wordt de code gestaged. Wanneer de develop branch voldoende is getest wordt
deze naar de master branch gepusht. Er moet namelijk altijd een werkende versie
staan op de master branch.

\subsubsection{Pull requests}
Wanneer het werk aan een taak klaar is, en de code getest is, zet degene die de
taak heeft uitgevoerd een pull request open. In deze pull request kunnen de andere
projectleden alle veranderingen inzien en zonodig commentaar geven om de kwaliteit
van de code te waarborgen.
Aan het einde van de sprint wordt er gezamenlijk gekeken naar alle code
en indien alles werkt zal de dev branch gemerged worden met de master branch.

\subsection{Unittesting}
In de eerste week van het project wordt er kort onderzoek gedaan naar mogelijke
unit testing frameworks, zoals Check, CUnit etc.

\subsection{Codestijl}
Om nette code op te kunnen leveren is er een codestijl opgezet voor zowel 
de code als de commit messages.

\subsubsection{Commits}
\begin{itemize}
    \item Er moet duidelijk worden aangegeven per commit wat er is aangepast / 
          toegevoegd.
    \item De stijl van commits is als volgt: in het Engels en in de tegenwoordige
          tijd.
\end{itemize}

\subsubsection{Code}
\textbf{Algemeen}

\begin{itemize}
  \item Alle files moeten eindigen met een newline.
  \item Namen moeten overeenkomen met de naamgevingen genoemd in 
      \texttt{naming\_conventions.txt}.
\end{itemize}

\textbf{C}
\begin{itemize}
  \item Bij inspringen (\texttt{TAB}) wordt er gebruik gemaakt van 4 spaties.
  \item Inclusion guards worden gebruikt om te voorkomen dat header files dubbel
        geïncludeerd worden. Hier een voorbeeld:
        \begin{lstlisting}[language=C,frame=single]
          #ifndef FILE_NAME_H
          #define FILE_NAME_H

          ...content...

          #endif /* FILE_NAME_H */
        \end{lstlisting}
  \item Waar mogelijk worden \texttt{enums} gebruikt in plaats van \texttt{defines}.
  \item Er worden standaard variabele types vanuit \texttt{<stdint.h>}.
  \item Bij pointers wordt de ster direct voor de variabele naam geplaatst
        bij het gebruik van een pointer.
  \item Gebruik \texttt{/* ... */} om meerdere lijnen als comment te zetten,
        \texttt{//} wordt gebruikt om enkele lijnen als comment te zetten.
  \item Als er nog specifieke zaken moeten gebeuren in een bepaald stuk code wordt
        dit aangegeven met \texttt{/* TODO: ... */}.
\end{itemize}
