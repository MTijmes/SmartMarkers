\section{Opdracht}
\label{sec:opdracht}
Smart Markers

Dit project is tot stand gekomen omdat er nog geen kadastrale registratiemogelijkheden 
zijn in ontwikkelingslanden. Het Kadaster kwam met het idee van \textit{smart markers}, 
deze markers verzamelen GPS-data om percelen inzichtelijk te maken. Het systeem moet 
robuust en zelfvoorzienend zijn, omdat de voorzieningen in ontwikkelingslanden erg 
uiteenlopen.

\subsection{Afbakening}
Aangezien er voor dit project een relatief korte tijdsduur in beslag genomen
kan worden, is het van groot belang om duidelijk af te bakenen wat er in deze
tijd gerealiseerd gaat worden.
Het project omvat slechts het opleveren van een prototype \textit{Smart Marker},
deze zal zeker nog niet de uiteindelijke compactheid hebben en ook slechts in 
kleine hoeveelheid opgeleverd worden.

\subsubsection{Binnen de scope}
\begin{itemize}
    \item LoRa radio module gebruik onderzoeken
    \item LoRa mesh netwerk onderzoeken en realiseren
    \item GPS-module gebruik onderzoeken
    \item GPS-(correctie)technieken toepassen
    \item Systeem documentatie ontwikkelen
    \item Mesh network op kleine schaal testen
\end{itemize}

\subsubsection{Buiten de scope}
\begin{itemize}
    \item Verschillende GPS-(correctie)technieken onderzoeken
    \item Doorontwikkeld product opleveren
    \item Mesh netwerk op ware schaal testen
    \item Afmetingen van het prototype optimaliseren
\end{itemize}

\newpage
\subsection{Voorwaarden voor succes}
Om dit project tot een succes te brengen is kennis van hardware, technieken
en programmeertalen benodigd. Hieronder staat de verzameling van verschillende
aspecten waarvan kennis benodigd is:
\subsubsection{Hardware}
\begin{itemize}
    \item STM32 NUCLEO-L073RZ development board
    \item RF-LORA-868-SO radio module
    \item u-blox EVK-7P GPS-module
\end{itemize}
\subsubsection{Technieken}
\begin{itemize}
    \item GPS
    \item DGPS
    \item LoRa
    \item Mesh networking
\end{itemize}
