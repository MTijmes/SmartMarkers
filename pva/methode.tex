\section{Planning en methode}

\subsection{Gebruikte methodiek}
Binnen dit project wordt gebruik gemaakt van de Scrum methode. Dit houdt
in dat er wordt gewerkt in sprints, deze sprints zijn elk twee
weken lang, wat inhoudt dat er in totaal vier sprints zullen zijn. Als
Scrum board wordt er gebruik gemaakt van de issueboard functionaliteit
in GitHub. Voor de documentatie wordt gebruik gemaakt van Latex aangezien
in het verleden is gebleken dat dit eenvoudig werkt en erg overzichtelijk is.

\subsection{Planning}
In de eerste week worden de project ondersteunende documenten gemaakt, als het
functioneel en technisch ontwerp. Vervolgens wordt er per sprint een planning
gemaakt.

Op 18 januari 2018 wordt het project tijdens de \textit{Winnovation} expositie
tentoongesteld.

\subsection{Activiteiten en producten}
Er moeten een aantal taken en verslagen worden voltooid in de komende
project periode. Dit betreft de volgende onderdelen:
\begin{itemize}
  	\item Functioneel ontwerp
  	\item Technisch ontwerp
    \item Voortgangsrapportages
	\item Prototype
\end{itemize}
Het functioneel en technisch ontwerp dienen af te zijn op het moment dat 
er begonnen wordt met programmeren. Op deze manier wordt er voorkomen 
dat er teveel werk wordt verricht of dat er verkeerde aannamens worden 
gedaan. De eerste sprint zal voornamelijk gericht zijn op het maken van 
het functioneel en technisch ontwerp zodat hierna geprogrammeerd kan
worden.

Het eind van iedere week wordt er een voortgangsrapportage naar de begeleider
gestuurd.

Het eindproduct, ofwel het prototype, wordt in de vierde sprint opgeleverd.
