\section{Planning en methode}

\subsection{Gebruikte methodiek}
Binnen dit project wordt gebruik gemaakt van de Scrum methode. Dit houdt
in dat er wordt gewerkt in sprints, deze sprints zijn elk twee
weken lang, wat inhoudt dat er in totaal drie sprints zullen zijn. Als
Scrum board wordt er gebruik gemaakt van de issueboard functionaliteit
in GitHub. Voor de documentatie wordt gebruik gemaakt van Latex aangezien
in het verleden is gebleken dat dit eenvoudig werkt en erg overzichtelijk is.


\subsection{Planning}
In de eerste week zal worden bekeken wat er allemaal gedaan moet worden en 
hoelang deze verschillende taken gaan duren. Vervolgens wordt er uit dit
overzicht een planning per sprint gemaakt.

% In de eerste week wordt bekeken wat er gedaan moet worden om het project
% te voltooien, hieruit wordt vervolgens per sprint een planning uit gemaakt.

\subsection{Activiteiten en producten}
Er moeten een aantal taken en verslagen worden voltooid in de komende
project periode. Dit betreft de volgende onderdelen:
\begin{itemize}
  	\item Functioneel ontwerp
  	\item Technisch ontwerp
	\item Het algoritme
	\item Lego robot
\end{itemize}
Het functioneel en technisch ontwerp dienen af te zijn op het moment dat 
er begonnen wordt met programmeren. Op deze manier wordt er voorkomen 
dat er teveel werk wordt verricht of dat er verkeerde aannamens worden 
gedaan. De eerste sprint zal voornamelijk gericht zijn op het maken van 
het functioneel- en technisch ontwerp zodat hierna geprogrammeerd kan
worden.

	
